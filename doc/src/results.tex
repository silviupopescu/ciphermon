% vim: set tw=78 sts=2 sw=2 ts=8 aw et ai:
The main goal, i.e. detecting malicious action being carried out by 
a suspiciuos application (namely angecrypt), of our PoC (Proof-of-Concept)
environment was achieved. We will further expand the process through
 which our result can be reproduced.

\subsection{Setup Tools and Projects}
In order to reproduce our result, a rooted phone is needed or at least an 
emulator that has superuser access. In our experiments, we resorted to 
using an emulator with Android 4.4.2 and API level 19 that was emulating 
an ARMv7 architecture. It is considered implicit the availability of an 
Android SDK as well as an Android NDK.

One would also need the sources of \emph{ADBI}\footnote{\url{https://github.com/crmulliner/adbi}}
and \emph{DDI}\footnote{\url{https://github.com/crmulliner/ddi}} 
projects developed by \cite{ddi}. In order to use our hook instead of the 
examples provided in DDI, our project\footnote{\url{https://github.com/silviupopescu/ciphermon/blob/master/jni/mon.c}} sources have to be dumped in the ddi/examples folder.

Last but not least the suspicious application is required. That can be 
found on Github as well under the name emph{angeapk}
\footnote{\url{https://github.com/cryptax/angeapk/tree/master/wrapping-apk/src/com/fortiguard/poc/angecrypt}}

\subsection{PoC Reproduction}

The suspiciuos APK can be found in the Github repository mentioned above. 
The \emph{adb} tool can be used to install the app on the phone/emulator.
Assuming your are in the angeapk folder:
\begin{verbatim}
adb install PocActivity-debug.apk
\end{verbatim}
will install the APK.

We will now compile binaries that support the hooking platform. We 
need to be in the ADBI folder.
\begin{verbatim}
cd adbi/hijack/jni
ndk-build
adb push ../libs/armeabi/hijack /data/local/tmp/
cd ../..
cd instruments/base/jni
ndk-build
cd ../../../..
\end{verbatim}
Next, we will compile the libraries needed for the dalvik machine. We 
need to be in the DDI folder.
\begin{verbatim}
cd ddi/dalvikhook/jni/libs
adb pull /system/lib/libdl.so
adb pull /system/lib/libdvm.so
cd ../
ndk-build
cd ../../..
\end{verbatim}

Before compiling the libraries needed for the hook, make sure there is 
a folder called ciphermon in ddi/examples that contains the jni folder 
with our custom hook.

\begin{verbatim}
cd ddi/examples/ciphermon/jni
ndk-build
\end{verbatim}

A library called \emph{libciphermon.so} should be available now. We 
just push it on the phone
\begin{verbatim}
adb push ../libs/armeabi/libciphermon.so /data/local/tmp
\end{verbatim}
and then run a shell within the phone/emulator and hook our library 
to the specified maliciuos application.
\begin{verbatim}
adb shell \# pentru conectat la dispozitiv Android
su
ps | grep ange \#to find the PID of said suspicious application (angecrypt)
cd /data/local/tmp
touch /data/local/tmp/ciphermon.log
chmod 777 /data/local/tmp/ciphermon.log
./hijack -d -p \$PID -l /data/local/tmp/libciphermon.so
cat /data/local/tmp/ciphermon/log
\end{verbatim}

The output from the last command should contain information 
pertaining the use of the targeted suspicious application.



