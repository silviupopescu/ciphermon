% vim: set tw=78 sts=2 sw=2 ts=8 aw et ai:

Since our research covers the Android operating systems, it is natural that
the proposed solution comes as an Android application. To this end, we have
made use of and repackaged Collin Mulliner's Dynamic Dalvik Instrumentation
framework. Full credit is given for his excellent implementation of Dalvik
hooking.

We have looked at the demonstrative application that researchers have
previously\cite{hiding-apk} developed in order to highlight encrypted malware.
While our approach is extensible, we have targeted this specific sample.

\subsection{Tools and Constraints}

The demonstrative application that highlights encrypted malware can be found 
on Github under the name
 \emph{angecrypt}\footnote{\url{https://github.com/cryptax/angeapk}}. 

We wrapped our solution around it so when we designed our hook we specifically 
targeted a function called \emph{doFinal()} \footnote{\url{ 
http://developer.android.com/reference/javax/crypto/Cipher.html\#doFinal()}}
by taking its input and output and comparing them to signatures of known 
file formats (in this particular demonstrative application, the input was a PNG
file and the output a valid APK file). 

We used the Android SDK\footnote{\url{https://developer.android.com/sdk/index.html\#Other}}
to build the Bootup Receiver, the New Application Receiver 
and the Main Application modules, and relied on the Android NDK\footnote{\url{https://developer.android.com/tools/sdk/ndk/index.html\#Downloads}}
when tackling the Hook Injector and the Hook modules.

One of the biggest constrain that we had was the fact that we used an 
emulator instead of a rooted Android phone due to financial limitations 
(warranty voiding due to rooting). Because of this we were unable to fully
 automate the Hook Injector and the hijacking of the above specified 
method had to be done manually.

